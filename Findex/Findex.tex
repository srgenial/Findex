% AER-Article.tex for AEA last revised 22 June 2011
\documentclass[]{AEA}

% The mathtime package uses a Times font instead of Computer Modern.
% Uncomment the line below if you wish to use the mathtime package:
%\usepackage[cmbold]{mathtime}
% Note that miktex, by default, configures the mathtime package to use commercial fonts
% which you may not have. If you would like to use mathtime but you are seeing error
% messages about missing fonts (mtex.pfb, mtsy.pfb, or rmtmi.pfb) then please see
% the technical support document at http://www.aeaweb.org/templates/technical_support.pdf
% for instructions on fixing this problem.

% Note: you may use either harvard or natbib (but not both) to provide a wider
% variety of citation commands than latex supports natively. See below.

% Uncomment the next line to use the natbib package with bibtex
\usepackage{natbib}

% Uncomment the next line to use the harvard package with bibtex
%\usepackage[abbr]{harvard}

% This command determines the leading (vertical space between lines) in draft mode
% with 1.5 corresponding to "double" spacing.
\draftSpacing{1.5}

\usepackage{hyperref}

\begin{document}

\title{Desarrollo Financiero y Crecmimiento Económico en México}

% \author{Author1 and Author2\thanks{Surname1: affiliation1, address1, email1.
% Surname2: affiliation2, address2, email2. Acknowledgements}}


\author{
  Mauricio Montiel, Tiare León\\
  Juan González, Juan Castro\thanks{
  Montiel, León: El Colegio de la Frontera Norte, \href{mailto:mmontiel.MEA2018@colef.mx, tiare.mea2018@colef.mx}{mmontiel.MEA2018@colef.mx, tiare.mea2018@colef.mx}.
  González, Castro: , \href{mailto:juangonzalez@colef.mx, jclimon.mea2018@colef.mx}{juangonzalez@colef.mx, jclimon.mea2018@colef.mx}.
}
}

\date{\today}
\pubMonth{06}
\pubYear{2019}
\pubVolume{}
\pubIssue{}
\JEL{O16, O47, C22}
\Keywords{Desarrollo Financiero, Crecimiento Económico, Series de tiempo}

\begin{abstract}
La hipótesis de McKinon-Shaw nos dice que las restricciones
gubernamentales sobre el sector financiero restringen y distorsionan el
proceso de desarrollo de este sector y, por ende, inhiben el proceso de
crecimiento económico. En este contexto, un sistema financiero
desregulado tiene efectos positivos en el crecimiento económico de largo
plazo. Por su parte, el sistema financiero mexicano ha pasado por dos
grandes etapas: de regulación hasta 1988 y posteriormente de
liberalización. El objetivo general de este trabajo es conitribuir a la
litaratura empírica al examinar los efectos de la intermediación
financiera en México de 1968 a 2018, lapso de 50 años en el que se han
llevado a cabo reformas importantes en el sector financiero tendientes a
liberalizarlo.
\end{abstract}


\maketitle

\section{Introducción}

La creciente competencia en el sector financiero por una mayor
rentabilidad y menor riesgo impulsa la expansión de los flujos
internacionales de capital procedentes de grandes bancos, empresas
trasnacionales y nuevos inversionistas institucionales que agrupan a un
gran numero de ahorradores de los países industrializados. Esto generó,
en los años setenta, los procesos de liberalización financiera (Tabares
and Daza 2000).

Esta liberalización se puede caracterizar como el proceso de otorgar al
mercado la autoridad para determinar quien obtiene y otorga el crédito y
a qué precio. Esta caracterización sugiere seis dimensiones para la
liberalización (Williamson and Mahar 1998): la eliminación de los
controles de crédito, la desregulación de los tipos de interés, la
entrada libre a la industria de servicios financieros, autonomía
bancaria, propiedad privada de los bancos y la liberación de los flujos
internacionales de capital.

Existen dos puntos de vista muy diferentes acerca de los efectos que
tiene la liberalización financiera. Desde un punto de vista, esta
fortalece el desarrollo financiero y contribuye al crecimiento económico
de largo plazo. Desde otro punto de vista, la liberalización financiera
provoca una toma excesiva de riesgos, aumenta la volatilidad
macroeconómica y conduce a crisis más frecuentes; sin embargo, aunque
esta liberalización conduce a crisis ocasionalmente, también promueve un
crecimiento económico promedio más rápido a largo plazo y, aunque estas
crisis sean costosas y tengan efectos recesivos, son eventos raros. Por
lo tanto, a largo plazo, los efectos positivos de la liberalización
financiera superan con creces los efectos perjudiciales de las crisis
(Ranciere, Tornell, and Westermann 2006).

Actualmente existe abundante evidencia de que el desarrollo financiero
derivado de la liberalización tiene un impacto positivo sobre el
crecimiento económico. Por un lado, la literatura empírica sugiere que
economías con un mejor funcionamiento del sector financiero obtienen un
mayor crecimiento económico.

Por otro lado, la literatura teórica ha identificado los mecanismos a
través de los cuales el desarrollo del sector financiero afecta a los
determinantes del crecimiento económico. En concreto, el sistema
financiero afecta positivamente al generar información sobre posibles
inversiones y asignar los recursos a las más eficientes, monitorear
proyectos de inversión y mejorar el gobierno corporativo, diversificar y
facilitar el manejo del riesgo, movilizar y reunir ahorros de distintos
individuos y facilitar el intercambio de bienes y servicios. Es
importante resaltar que el sistema financiero también incentiva la
acumulación de capital físico y humano, la innovación tecnológica y el
flujo internacional de capitales \textbf{{[}cermeno2016desarrollo{]}}

Cabe destacar que este flujo internacional de capitales existía incluso
antes de que se diera el proceso de liberalización financiera; la
diferencia radica en que antes los gobiernos frenaban el acceso a nuevos
agentes en el sector financiero. En México; sin embargo, incluso después
de la liberalización financiera, el ingreso neto de capitales
internacionales no se ha reflejado en un mayor crecimiento económico,
pues este se ha mantenido prácticamente estancado (Tabares and Daza
2000).

El objetivo general de este trabajo es conitribuir a la litaratura
empírica al examinar los efectos de la intermediación financiera en
México de 1968 a 2018, lapso de 50 años en el que se han llevado a cabo
reformas importantes en el sector financiero tendientes a liberalizarlo.
Verificar que si el desarrollo financiero se estanca entonces el
crecimiento económico será menor que el esperado. Además se busca
examinar, a través de la creación de un índice, si la regulación
financiera inhibió el desarrollo de las instituciones intermediarias y
entender su comportamiento antes y después de la liberalización
financiera.

La revisión de estos efectos a largo plazo y la creación de el índice se
elaborarán con base en las metodologías de series de tiempo. En primer
lugar se determinará el orden de integración de las series utilizando la
pruabde Dickey-Fuller aumentada; en seguida se procederá a identificar
las relaciones de largo plazo entre las variables con base en la
metodología de Johansen (1988). Se realizarán pruebas de causalidad y
para medir el desarrollo financiero se construirá un índice utilizando
la metodología de componentes principales.

Esta investigación se realizará para México en un periodo de 50 años que
abarcará desde 1968 hasta 2018. El documento presenta la siguinte
estructura. En la pimera sección se discutirá la teoría y la evidencia
empírica que sustentan la relación entre desarrollo financiero y
crecimiento económico. En la segunda sección se revisará el desarrollo
histórico y el estado actual del sistema financiero en nuestro país y en
la tercera sección sedescribirá el modelo empírico a utilizar y se
desarrollará la metodología econométrica. Por último, habrá una sección
en donde se expongan las conclusiones y se resalten los principales
hallazgos de la investigación. En esta sección se discutirá si las
variables fueron adecuadas para explicar la problemática y se expondrá
el cumplimiento de las respectivas hipótesis.

\section{Revisión teórica y empírica}

De acuerdo a la teoría, la principal función de los intermediarios
financieros es canalizar de manera apropiada el ahorro de los
consumidores hacia el gobierno y los empresarios que piden prestamos.

En este sentido, el sistema bancario debe aminorar las asimetrías de la
información entre los prestatarios y los prestamistas, así como asegurar
que los fondos se vayan a actividades productivas que aseguren la
eficiencia económica y con ella el crecimiento. Los modelos de
crecimiento económico neoclásicos indican que la condición para el
crecimiento a largo plazo es el crecimiento de la fuerza laboral y el
progreso técnico (Solow, 1956). En este sentido, desde el punto de vista
neoclásico, la intermediación financiera solo influiría en el
crecimiento a través de la canalización del ahorro hacia ese progreso
técnico.

En la teoría de crecimiento endógeno la intermediación financiera
influye en el crecimiento a través de la inversión en Investigación y
Desarrollo (Romer 1986). Por lo tanto, el punto principal es que en
última instancia la intermediación financiera aumentará el ahorro que
será utilizado para invertir y aunque algunas de estas teorías
desarrollen modelos analíticos para entender la relación entre el
desarrollo financiero y el crecimiento económico solamente han sido
anecdóticas (Fase and Abma 2003) y las pruebas empíricas de esa relación
son un campo de estudio aparte. Por lo tanto, esta investigación estará
enfocada en el diseño estadístico para la validación empírica para el
caso de México.

Con respecto a esta validación empírica existen trabajos como el de
Goldsmith (1969) en el cual analizó el desarrollo financiero. Utilizó
datos anuales de 35 países más o menos homogéneos y logró documentar el
desarrollo de intermediarios financieros para explicar el crecimiento.
Sus resultados apoyan la opinión de que el crecimiento económico y el
nivel de desarrollo financiero están correlacionados. Encontró esta
correlación positiva entre el desarrollo financiero y económico para un
gran número de países.

Por otro lado, existen trabajos como el de King and Levine (1993) que en
el contexto de los modelos endógenos estudiaron 77 países durante el
período 1960--1989 para ver la relación entre el desarrollo financiero y
el crecimiento económico. Ellos tuvieron en cuenta otros factores que
pueden influir en el crecimiento económico además de la estructura
financiera. Sus resultados muestran una relación fuerte y positiva y
descubrieron que el desarrollo financiero es un buen indicador del
crecimiento de largo plazo. King y Levine también hicieron un análisis
causal entre el desarrollo financiero y el crecimiento económico. Sus
resultados sugieren que inicialmente, el nivel de desarrollo financiero
predice el crecimiento económico.

\section{Reformas financieras en México}

Partiendo de la hipótesis de que las restricciones gubernamentales sobre
el sistema financiero restringen y distorsionan el proceso de desarrollo
financiero y, en consecuencia, reducen el crecimiento de la economía
\textbf{{[}rodriguez2009desarrollo{]}}, es posible analizar estos
efectos sobre la economía Méxicana.

México alcanzó altas tasas de crecimiento económico con inflación baja
durante los años sesenta, pero a mediados de la década siguiente la
inestabilidad internacional en el mercado petrolero originó la
devaluación de la moneda, la fuga de capitales y finalmente la crisis de
la deuda de 1982, agravada por las dificultades para el acceso al
crédito internacional lo que a su vez dificultó el proceso de desarrollo
financiero.

Debido a que el Estado era el principal impulsor de la economía, su
gasto creció hasta entrar en déficit y tuvo que recurrir al
endeudamiento externo debido a que el financiamiento interno se
encontraba reprimido. Esta represión se manifestaba a través la
regulación de las tasas de interés, la existencia de controles
cuantitativos del crédito, el uso de reservas obligatorias por parte de
la banca comercial para el otorgamiento de crédito al gobierno, etc.

En el sexenio del presidente Carlos Salinas (1989-1994), se argumentaba
que la reforma financiera era necesaria ya que se había desarrollado un
marco institucional incapaz de responder a los choques externos. En
consecuencia, el gobierno transfirió a la iniciativa privada la tarea de
impulsar la economía, incluyendo el financiamiento para las actividades
productivas mediante el nuevo marco regulatorio de liberalización y
apertura del sistema financiero. Lo que se esperaba era promover el
crecimiento económico a través del desarrollo financiero.

El inicio de la década de 1990 se caracterizó fundamentalmente por el
proceso de liberalización del sistema financiero mexicano; sin embargo,
a principios de esa década, existió una gran cantidad de choques
negativos externos e internos que culminaron en la crisis financiera y
de balanza de pagos a finales de 1994. Una vez superada esta crisis, el
gobierno comenzó a reestructurar el sistema bancario y a rescatar a los
deudores de la banca y permitió la inversión extranjera en la banca
comercial doméstica \textbf{{[}rodriguez2009desarrollo{]}}.

\section{Análisis econométrico}

Para analizar empíricamente el efecto que tiene el desarrollo financiero
sobre el crecimiento económico de México se propone la siguiente función
que incluye variables económicas fundamentales de caracter estructural
como el PIB per cápitam denotado por \(Y_{t}\), el Índice de Formación
Bruta de Capital Fijo denotado como FBCF y el Índice de Desarrollo
Financiero denotado como FINDEX

\begin{align}
  \Delta \ln Y_{t}=\alpha FINDEX + \beta FBCF + \varepsilon _{t}
\end{align}

El símbolo \(\Delta\) es el operador de primera diferencia y \(\ln\)
indica que los datos están en logarítmos. \(\varepsilon _{t}\) es un
témino de residuales con media cero y varianza constante. De acuerdo
esta ecuación se espera que que el coeficiente \(\alpha\) de la variable
de desarrollo financiero sea positivo y significativo al igual que el
coeficiente \(\beta\).

Estas relaciones se estudian con base en la metodología de series de
tiempo. Como primer paso se determinará el orden de integración de las
series económicas con la prueba de Dikey-Fuller aumentada y en seguida
se procederá a examinar la relación a largo plazo entre ellas.

En análisis a largo plazo se analiza a partir del concepto de
cointegración que Engel y Granger (1988) definieron de la siguiente
manera: Un conjunto de variables económicas está cointegado si para un
vecor de coeficientes
\(\beta = \left (\beta_{1} , \beta _{2},...\beta _{n} \right )\)
denominado vector de cointegración y un vector de variables
\(Y_{t} = \left (Y_{1t} , Y _{2t},...Y_{nt} \right )\) existe una
relación tal que:

\begin{align}
  \beta Y_{t}=\varepsilon_{t}
\end{align}

donde \(\varepsilon_{t}\) tiene que ser un proceso estacionario. El
requisito para esta cointegración es que las variables del vector
\(Y_{y}\) deben ser del mismo orden de integración y que su combinación
lineal sea de un orden de integración menor. El procedimiento estandar
de cointegración es como sigue: a) se realizan pruebas de
estacionareidad a todas las variables involucradas en el modelo para
deteminar su orden de integración esperando que todos tengan el mismo y
b) se estima la ecuación de cointegración para después analizar sus
residuales y comprobar que sean estacionarios. Si los residuos son
estacinarios podemos concluir que nuestras variables son cointegradas.

Sin embargo, esta técnica no permite analizar mas de dos variables y es
por esa razón que se opto por la metodología de Johansen (1988), la cual
nos permite identificar si existe una relación a largo plazo entre dos o
más variables.

\hypertarget{descripcion-de-los-datos}{%
\subsection{Descripción de los datos}\label{descripcion-de-los-datos}}

En el estudio de King and Levine (1993) los autores utilizaron cuatro
medidas para describir el nivel de desarrollo financiero. La primera
medida es de los pasivos líquidos del sistema financiero. La segunda
medida describe la importancia de los bancos para la asignación de
crédito. La tercera y cuarta medida representan, de dos maneras
diferentes, el monto de los préstamos a empresas privadas.

Goldsmith (1969) por su parte, propone un indicador que consiste en la
relación entre los activos de los intermediarios financieros y el
Producto Nacional Bruto (PNB).

Las variables que nos indica la teoría son el crédito bancario al sector
privado con respecto al PIB, los activos bancarios comerciales con
respecto a los activos totales, el M2 con respeto al PIB, los activos de
la banca central con respecto a los activos totales y los pasivos
líquidos con respecto al PIB (Levine 1997).

En México, las variables que se pueden utilizar para analizar el
desarrollo financiero son:

\begin{itemize}
\item Liquid liabilities to GDP %
\item Private credit by deposit money banks and other financial institutions to GDP %
\item Domestic credit provided by financial sector % of GDP
\item Domestic credit to private sector % of GDP
\item Masa Monetaria con respecto al PIB
\end{itemize}

Estas variables se obtuvieron de los \emph{World Development Indicators}
del Banco Mundial. \textbf{HAY QUE TRADUCIRLAS CORRECTAMENTE}

Ang and McKibbin (2007) indican que muchas variables financieras están
altamente correlacionadas; sin embargo, separadas no servirían para
explicar el grado de desarrollo financiero. Una opición es la
construcción de un índice mediante la técnica de componentes
principales, que consiste en obtener una serie de datos que proviene de
un conjunto de variabls correlacionadas a través de la obtención de
valores propios con los cuales puede hacer un promedio ponderado y
obtener dicha serie.

La tabla 1 contiene los resultados del análisis de componentes
principales en donde se puede observar que el primer componente explica
el 59.46\% de la varianza de los datos. Para construir el índice
extraemos las contribuciones individuales de las variables LLG
(24.84\%),PCDM (25.16\%), DCFS (2.63\%), DCPS (25.27\%), M2 (23.07\%)
del vector 1 las cuales se utilzan para construir una serie ponderada
que se mestra en la figura XX.

\begin{table}

\caption{Análisis deComponentes Principales}

\begin{tabular}{llllll}
& CP1 & CP2 & CP3 & CP4 & CP5 \\
Eigenvalor & 2.97 & 1.25 & 0.61 & 0.13 & 0.02 \\
Porcentaje de varianza & 59.46 & 25.06 & 12.30 & 2.74 & 0.42 \\
Porcentaje acumulado & 59.46 & 84.52 & 96.83 & 99.57 & 100 \\
*Variable* & Vector 1 & Vector 2 & Vector 3 & Vector 4 & Vector 5 \\
LLG & 23.84 & 0.21 & 42.36 & 17.68 & 15.88 \\
PCDM & 25.16 & 4 & 5 & 4 & 5 \\
DCFS & 2.63 & 4 & 5 & 4 & 5 \\
DCPS & 25.27 & 4 & 5 & 4 & 5 \\
M2 & 23.07 & 4 & 5 & 4 & 5
\end{tabular}

\begin{tablenotes} [Nota]
es mucho trabajo hacer esta tabla, la dejaré para después.
\end{tablenotes}

\begin{tablenotes}[Fuente]
elaboración propia.
\end{tablenotes}

\end{table}

\begin{figure}
Figure XX.
\caption{Caption for figure below.}
\begin{figurenotes}
Figure notes without optional leadin.
\end{figurenotes}
\begin{figurenotes}[Source]
Figure notes with optional leadin (Source, in this case).
\end{figurenotes}
\end{figure}

\hypertarget{analisis-empirico}{%
\subsection{Análisis Empírico}\label{analisis-empirico}}

\textbf{AQUÍ VA TODO EL ANÁLISIS Y GRÁFICOS. YA TENGO LOS RESULTADOS EN
UN R-NOTEBOOK}

\bibliographystyle{apa}
\bibliography{references}

\section{Bibliografía}

\hypertarget{refs}{}
\leavevmode\hypertarget{ref-ang2007financial}{}%
Ang, James B, and Warwick J McKibbin. 2007. ``Financial Liberalization,
Financial Sector Development and Growth: Evidence from Malaysia.''
\emph{Journal of Development Economics} 84 (1). Elsevier: 215--33.

\leavevmode\hypertarget{ref-fase2003financial}{}%
Fase, Martin MG, and RCN Abma. 2003. ``Financial Environment and
Economic Growth in Selected Asian Countries.'' \emph{Journal of Asian
Economics} 14 (1). Elsevier: 11--21.

\leavevmode\hypertarget{ref-goldsmith1969financial}{}%
Goldsmith, Raymond William. 1969. ``Financial Structure and
Development.''

\leavevmode\hypertarget{ref-johansen1988statistical}{}%
Johansen, Søren. 1988. ``Statistical Analysis of Cointegration
Vectors.'' \emph{Journal of Economic Dynamics and Control} 12 (2-3).
Elsevier: 231--54.

\leavevmode\hypertarget{ref-king1993finance}{}%
King, Robert G, and Ross Levine. 1993. ``Finance and Growth: Schumpeter
Might Be Right.'' \emph{The Quarterly Journal of Economics} 108 (3). MIT
Press: 717--37.

\leavevmode\hypertarget{ref-levine1997desarrollo}{}%
Levine, Ross. 1997. ``Desarrollo Financiero Y Crecimiento Económico:
Enfoques Y Temario.'' \emph{Journal of Economic Literature} 35:
688--726.

\leavevmode\hypertarget{ref-ranciere2006decomposing}{}%
Ranciere, Romain, Aaron Tornell, and Frank Westermann. 2006.
``Decomposing the Effects of Financial Liberalization: Crises Vs.
Growth.'' \emph{Journal of Banking \& Finance} 30 (12). Elsevier:
3331--48.

\leavevmode\hypertarget{ref-romer1986increasing}{}%
Romer, Paul M. 1986. ``Increasing Returns and Long-Run Growth.''
\emph{Journal of Political Economy} 94 (5). The University of Chicago
Press: 1002--37.

\leavevmode\hypertarget{ref-tabares2000desregulacion}{}%
Tabares, Ramón Sánchez, and Alfredo Sánchez Daza. 2000. ``Desregulación
Y Apertura Del Sector Financiero Mexicano.'' \emph{Comercio Exterior} 50
(8): 690.

\leavevmode\hypertarget{ref-williamson1998survey}{}%
Williamson, John, and Molly Mahar. 1998. \emph{A Survey of Financial
Liberalization}. International Finance Section, Department of Economics,
Princeton University.

\end{document}

